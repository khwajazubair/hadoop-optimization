% Chapter Template

\chapter{Experimental Environment} % Main chapter title

\label{Chapter 3} % Change X to a consecutive number; for referencing this chapter elsewhere, use \ref{ChapterX}

\lhead{Chapter 3. \emph{Experimental Environment}} % Change X to a consecutive number; this is for the header on each page - perhaps a shortened title

%----------------------------------------------------------------------------------------
%	SECTION 1
%----------------------------------------------------------------------------------------

\section{Introduction}

All the experiments are tested in simulated environment.The environment consist of physical resources,virtual operating system,hadoop software,open stack and tools/scripts to process and analyse the log data.
To run the experiments , a set of Ubuntu 12.10 Virtual machine is spawned and installed on top of physical computers. Then, the process of hadoop-snap-shot-3 installation master and tata node configuration happens. After completion of hadoop setup, using terasort workload is generated and stored on datanodes.The masternode does not act as data node in our experiments. Once, the workload generation is completed, the terasort starts process the data using hadoop's default capacity scheduler. The openstack software is used to manage the VMs. I also use scripts to process and analyze the logs(see appendix).  

%-----------------------------------
%	SUBSECTION 1
%-----------------------------------
\subsection{Physical Resources}
A total number of seven(7) computer machines connected through central switch is used to run the experiments.All the computers are connected using Gigabit ethernet port to the switch. Each computer has sixteen(16)GB of RAM(Random Access Memory).The computers are equipped witch eight(8) CPU(Central Processing Unit), where the speed of each CPU is aproximately 2,3 GHZ.The system uses 10 GB of disk space to store the virtual machine and hadoop software. Additional mounted hard disk space of seventy two(72) GB is provided as NFS(Network File System) storage to each computer. 

%-----------------------------------
%	SUBSECTION 2
%-----------------------------------

\subsection{Terasort}

\textbf{Teragen} - Generates the random data that is used as input data for terasort.The data is generated in rows and the format of 
row is "<10 bytes key><10 bytes rowid><78 bytes filler>". 
The keys are random characters from the set ‘ ‘ .. ‘~’, rowid is justified row id as a int and the filler consists of 7 runs of 10 characters from ‘A’ to ‘Z’.Teragen divides the number of rows by the desired number of tasks and assigns set of rows to each map.\ref{TeraByte Sort on Apache Hadoop Owen O’MalleyYahoo!}


\textbf{TeraSort} - It is implemented as a MapReduce sort job with a custom partitioner that uses a sorted list of n-1 sampled keys that define the key range for each reduce.


%----------------------------------------------------------------------------------------
%	SECTION 2
%----------------------------------------------------------------------------------------

\section{Main Section 2}

Sed ullamcorper quam eu nisl interdum at interdum enim egestas. Aliquam placerat justo sed lectus lobortis ut porta nisl porttitor. Vestibulum mi dolor, lacinia molestie gravida at, tempus vitae ligula. Donec eget quam sapien, in viverra eros. Donec pellentesque justo a massa fringilla non vestibulum metus vestibulum. Vestibulum in orci quis felis tempor lacinia. Vivamus ornare ultrices facilisis. Ut hendrerit volutpat vulputate. Morbi condimentum venenatis augue, id porta ipsum vulputate in. Curabitur luctus tempus justo. Vestibulum risus lectus, adipiscing nec condimentum quis, condimentum nec nisl. Aliquam dictum sagittis velit sed iaculis. Morbi tristique augue sit amet nulla pulvinar id facilisis ligula mollis. Nam elit libero, tincidunt ut aliquam at, molestie in quam. Aenean rhoncus vehicula hendrerit.